%%%%%%%%%%%%%%%%%%%%%%%%%%%%%%%%%%%%%%%%%%%%%%%%%%%%%%%%%%%%%%%%%%%%%%%%%%%%%%%
%                                       __                                    %
%                     ____  ____ ______/ /______ _____ ____                   %
%                    / __ \/ __ `/ ___/ //_/ __ `/ __ `/ _ \                  %
%                   / /_/ / /_/ / /__/ ,< / /_/ / /_/ /  __/                  %
%                  / .___/\__,_/\___/_/|_|\__,_/\__, /\___/                   %
%                 /_/                          /____/                         %
%                                                                             %
%%%%%%%%%%%%%%%%%%%%%%%%%%%%%%%%%%%%%%%%%%%%%%%%%%%%%%%%%%%%%%%%%%%%%%%%%%%%%%%

%% use os archlinux
%% TEX Wiki https://texwiki.texjp.org/

\usepackage[dvipdfmx]{graphicx}
% graphicx:画像を挿入したり,テキストや図の拡大縮小・回転を行うためのパッケージ.
\usepackage{subcaption}
\usepackage{float}
% float:figureのオプション[H]が使えるようになる.
\usepackage[unicode, hidelinks, pdfusetitle, setpagesize=false]{hyperref}
% hyperref:リンクを使用するためのパッケージ.
\usepackage{xcolor, textcomp, bm, wrapfig, enumerate, cite, tcolorbox}
% xcolor:文字に色.
% textcomp:使える記号が増える.
% 記号 http://www.yamamo10.jp/yamamoto/comp/latex/make_doc/symbol/index.php
% bm:ベクトル表現.
% wrapfig:図にテキストを回り込ませる.
% enumerate:箇条書き.
% cite:\citeの記法変更.
% tcolorbox:フレームのある枠の作成.
\usepackage{amsmath, amssymb}
% amsmath, amssymb:数式環境.
\usepackage{here}
% here:\begin{figure}[H]と記述することで好きな位置に図を挿入する.
